%% INTRODUCTION


\subsection{Domain overview}
Flame detectors are widely used in areas subject to fire hazards such as oil and gas installations and chemical plants. Detectors consist of hardware and software subject to SIL (Security Integrity Level) certification, aiming and reducing risk effects. Standards such as the IEC 61508\cite{wiki:IEC61508} are used to certify equipment and ensure compliance.  

\subsection{Problems to be tackled}

To carry out the analysis the barriers to overcome consist of:

\begin{itemize}
\item processing log files
\item choosing appropriate libraries
\item engineer features to inform our decisions

\end{itemize}

\subsection{Analytical questions}

The analytical questions we want to ask is how can signal be differentiated from noise? Though distance functions? Linear regression? A combination of both?

\subsection{Objectives}

\begin{itemize}
\item Trial different techniques
\item Find a suitable model to eliminate noise (smoothing algorithm)

Note, this could be a combination of existing algorithms, to make best use of the attributes present in available dataset.
\end{itemize}

\subsection{Data source} 

The data being analysed consists of data logs generated by commercially available flame detectors. Twenty eight files were examined in total, generated under test conditions. One log file (Test45.log) containing real fire data, some of the remaining logs containing RF (radio frequency) data erroneously reported in software as fire - this will be shown in attributes.

The logs files were obtained from a flame detector undergoing Functional Safety of Electrical/Electronic/Programmable Electronic Safety-related Systems, as defined in the IEC 61508 standard \cite{wiki:IEC61508}. One log file (Test45.log) containing real fire data, the other logs, some containing RF (radio frequency) data erroneously reported in software as fire.

\subsection{Analysis strategy}

Once our signal and noise have been characterized, our analysis strategy consists of engineering features, as well as creating models to quantify the levels of noise and signal. Distance functions, fft transforms and filters, such as proposed by Savitzky and Golay \cite{Savitzky:1964} provide a scheme to smooth signals, eliminating noise. In this work, we shall examine such schemes and compare the end results, determining by sch observations a filtering strategy to eliminate noise generated by GHz frequency interference.
